%=================================================================%

\section{Stats}

% Pandas provides nifty methods to understand your data. 
%I am highlighting the describe, correlation, covariance, and correlation methods that I use to quickly make sense of my data.

%=================================================================%

\subsection{describe}

The describe method provides quick stats on all suitable columns.

%=================================================================%
\begin{framed}
\begin{verbatim}
df.describe()

            col1      col2
count    4.00000  5.000000
mean     2.65000  3.200000
std      4.96689  3.701351
min      0.10000 -1.000000
25%      0.17500  1.000000
50%      0.20000  2.000000
75%      2.67500  6.000000
max     10.10000  8.000000
\end{verbatim}
\end{framed}

%=================================================================%
\subsection{covariance}

The \texttt{cov} method provides the covariance between suitable columns.


\begin{framed}
\begin{verbatim}
>>> df.cov()

                 col1       col2
    col1    24.670000  12.483333
    col2    12.483333  13.700000
\end{verbatim}
\end{framed}

%=================================================================%

\subsection{correlation}

The \texttt{corr} method provides the correlation between suitable columns.
\begin{framed}
	\begin{verbatim}
>>> df.corr()
  
                 col1       col2
    col1     1.000000   0.760678
    col2     0.760678   1.000000
\end{verbatim}
\end{framed}

%=================================================================%
\end{document}